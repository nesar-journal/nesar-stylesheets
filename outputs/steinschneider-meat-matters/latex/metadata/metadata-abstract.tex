In this article, I explore the processes through which Tamil-based Śaivism came to be conceptually equated with the maintenance of a vegetarian diet, a development reflected in the modern Tamil word \emph{caivam} (Skt. \emph{śaiva}), which in colloquial speech primarily signifies lacto-vegetarian cuisine. I contend that although Tamil Śaiva literary sources have long articulated the normativity of vegetarianism, the conflation of Śaiva praxis with plant-based dietary habits likely dates to the late sixteenth and seventeenth centuries. Such, at least, is the picture that emerges from a consideration of the \emph{Kolaimaṟuttal} (\emph{Rejecting Killing}), a brief polemic against animal slaughter likely composed in the then-frontier region of what is now the suburbs of Coimbatore, which emphasizes dietary nonviolence as the quintessential Śaiva virtue and the principal basis for demarcating Śaivism from other religions. A close reading of this hitherto unstudied text suggests that early modern Tamil Śaiva food discourse transformed, at least in part, in response to the emergence of new notions of “self” and “other” in this period, which prompted a corresponding need to rethink the contour and configuration of community boundaries.